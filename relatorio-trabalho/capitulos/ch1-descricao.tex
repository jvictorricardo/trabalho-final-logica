% ==============================================================================
% PG - Nome do Aluno
% Capítulo 1 - Introdução
% ==============================================================================
\chapter{Descrição}
\label{sec-desc}

\par Mario Odyssey foi a 7ª aventura (e última até o momento) de Mario, sendo lançado em 2017. Mario Odyssey segue o mesmo esquema de história de títulos anteriores e presta certa homenagem a eles, em especial ao Mario 64, tendo um reino inteiro dedicado ao título do Nintendo 64 que deu modelo aos jogos de plataforma 3D. Assim como todos os jogos do Mario, Odyssey começa com a princesa Peach sendo raptada mais uma vez por Bowser, que derrota Mario e o derruba de sua aeronave. Após isso, ele acorda em um reino desconhecido (Cap Kingdom) e se encontra com Cappy (personagem que vai ser seu parceiro durante toda a sua jornada), que teve sua irmã Tiara raptada por Bowser.
\par Unidos pelo objetivo de derrotar Bowser e resgatar seus amigos, Cap e Mario derrotam um dos capangas de Bowser e vão até a nave Odyssey, restaurando-a com a ajuda das Power Moons, o que acontecerá com frequência durante sua passagem pelos reinos. São 14 reinos que Mario e Cappy percorrem durante sua aventura, sendo eles: Cap Kingdom, Cascade Kingdom, Sand Kingdom, Wooded Kingdom, Lake Kingdom, Cloud Kingdom, Lost Kingdom, Metro Kingdom, Seaside Kingdom, Snow Kingdom, Luncheon Kingdom, Ruined Kingdom, Bowser's Kingdom, Moon Kingdom. Além desses, temos mais 3 reinos adicionais que são jogáveis após o encerramento da história: Mushroom Kingdom, Dark Side e Darker Side. Na maioria dos reinos, Mario os visita com o objetivo de perseguir Bowser, enquanto Bowser visita os reinos para roubar algo para seu casamento (e.g. no Cap Kingdom ele rouba Tiara, e no Wooded Kingdom ele rouba as flores para seu casamento). 
\par Uma das principais novidades de Odyssey foi o sistema de capturas que ocorrem quando Mario joga Cappy na cabeça de boa parte dos personagens, passando a ‘possuir’ aquele personagem e adquirindo as habilidades do mesmo, que o ajudam a pular obstáculos que são impossíveis para Mario, voar pelo cenário, atirar projéteis e etc. variando de acordo com o tipo de captura realizada.
O presente trabalho tem por objetivo principal modelar, por meio de um programa escrito em Prolog, alguns aspectos do mundo de Mario Odyssey, apresentando seus reinos, alguns fatos da história e personagens.
